
\documentclass[twoside]{article}
\usepackage[colorlinks,linkcolor=black]{hyperref}
\usepackage{xeCJK}
\usepackage{fancyhdr}
\usepackage{amsmath, amsthm}
\usepackage{listings,xcolor}
\usepackage{geometry}
\usepackage{fontspec}
\newcommand{\HRule}{\rule{\linewidth}{0.5mm}}
\geometry{left=2.5cm,right=2.5cm,top=2.5cm,bottom=2.5cm}
\lstset{
    language    = c++,
    breaklines  = true,
    captionpos  = b,
    tabsize     = 4,
    numbers     = left,
    columns     = fullflexible,
    keepspaces  = true,
    commentstyle = \color[RGB]{0,128,0},
    keywordstyle = \color[RGB]{0,0,255},
    basicstyle   = \small\ttfamily,
    rulesepcolor = \color{red!20!green!20!blue!20},
    showstringspaces = false,
}
\title{badcw's template}
\author{So Like Coding? You Baldy}
\pagestyle{fancy}
\fancyhf{}
\fancyhead[C]{badcw's template, So Like Coding? You Baldy}
\begin{document}\small
\begin{titlepage}
\begin{center}
\vspace*{0.5cm}\includegraphics[width=0.75\textwidth]{logo.jpg} \\ [2cm]
\HRule \\ [1cm]
\textbf{\Huge{badcw's template}} \\ [0.5cm]
\HRule \\ [4cm]
\textbf{\Huge{WUST}} \\ [1cm]
\LARGE{So Like Coding? You Baldy}
\vfill
\Large{\today}
\end{center}
\clearpage
\end{titlepage}
\tableofcontents\clearpage
\pagestyle{fancy}
\lfoot{}
\cfoot{\thepage}\rfoot{}
\setcounter{section}{-1}
\setcounter{page}{1}
\clearpage\section{2019New}
\subsection{2019New}
\subsubsection{mint}
\begin{lstlisting}
const int mod = 998244353;

struct mint {
  int n;
  mint(int n_ = 0) : n(n_) {}
};

mint operator+(mint a, mint b) { return (a.n += b.n) >= mod ? a.n - mod : a.n; }
mint operator-(mint a, mint b) { return (a.n -= b.n) < 0 ? a.n + mod : a.n; }
mint operator*(mint a, mint b) { return 1LL * a.n * b.n % mod; }
mint &operator+=(mint &a, mint b) { return a = a + b; }
mint &operator-=(mint &a, mint b) { return a = a - b; }
mint &operator*=(mint &a, mint b) { return a = a * b; }
ostream &operator<<(ostream &o, mint a) { return o << a.n; }

mint dp[20][2][2][2][1 << 10];
mint dp2[20][2][2][2][1 << 10];\end{lstlisting}
\subsubsection{不重叠区间贪心}
\begin{lstlisting}
#include <bits/stdc++.h>
#define ll long long
using namespace std;

const int maxn = 5e5+5;
pair<int, int> a[maxn];
int main() {
    int n;
    cin >> n;
    for (int i = 1; i <= n; ++i) {
        cin >> a[i].second >> a[i].first;
    }
    sort(a + 1, a + 1 + n);
    int res = 1;
    int tmp = a[1].first;
//    printf("%d %d\n", a[1].second, a[1].first);
    for (int i = 2; i <= n; ++i) {
        if (a[i].second > tmp) {
            res ++;
//            printf("%d %d\n", a[i].second, a[i].first);
            tmp = a[i].first;
        }
    }
    printf("%d\n", res);
    return 0;
}

\end{lstlisting}
\subsubsection{主席树区间k大}
\begin{lstlisting}
/*************************************************************************
	> File Name: a.cpp
	> Author: badcw
	> Mail: 952223482@qq.com
	> Created Time: 2018年07月21日 星期六 08时47分54秒
 ************************************************************************/

#include <bits/stdc++.h>
#define ll long long
using namespace std;

const int maxn = 100005;
int n, m;
int a[maxn];
int root[maxn];
int cnt = 0;
vector<int> b;
struct node {
    int l, r, val;
}p[maxn * 40];

void update(int l, int r, int pre, int &now, int pos) {
    now = ++cnt;
    p[now] = p[pre];
    p[now].val++;
    if (l == r) {
        return;
    }
    int mid = l + r >> 1;
    if (pos <= mid) update(l, mid, p[pre].l, p[now].l, pos);
    else update(mid + 1, r, p[pre].r, p[now].r, pos);
}

int query(int l, int r, int x, int y, int k) {
    if (l == r) return b[l - 1];
    int mid = l + r >> 1;
    int temp = p[p[y].l].val - p[p[x].l].val;
    if (k <= temp) return query(l, mid, p[x].l, p[y].l, k);
    return query(mid + 1, r, p[x].r, p[y].r, k - temp);
}

int main(int argc,char *argv[])
{
    while (scanf("%d%d", &n, &m) != EOF) {
        b.clear();
        cnt = 0;
        for (int i = 1; i <= n; ++i) scanf("%d", &a[i]), b.push_back(a[i]);
        sort(b.begin(), b.end());
        b.erase(unique(b.begin(), b.end()), b.end());
        for (int i = 1; i <= n; ++i) {
            update(1, b.size(), root[i - 1], root[i], lower_bound(b.begin(), b.end(), a[i]) - b.begin() + 1);
        }
        int L, R, k;
        while (m--) {
            scanf("%d%d%d", &L, &R, &k);
            printf("%d\n", query(1, b.size(), root[L - 1], root[R], k));
        }
    }
    return 0;
}\end{lstlisting}
\subsubsection{莫队}
\begin{lstlisting}
#include <bits/stdc++.h>
#define ll long long
using namespace std;
const int maxn = 200005;

struct MO {
    int l, r, id;
}q[maxn];

int n, m, col[maxn], block, belong[maxn];
ll vis[maxn * 10], ans;
ll res[maxn];
bool cmp(const MO& a, const MO& b) { return belong[a.l] == belong[b.l] ? a.r < b.r : a.l < b.l; }
void add(ll x) {
    vis[x] ++;
    ans += x * (vis[x] * vis[x] - (vis[x] - 1) * (vis[x] - 1));
}

void del(ll x) {
    vis[x] --;
    ans -= x * ((vis[x] + 1) * (vis[x] + 1) - vis[x] * vis[x]);
}

int main() {
    scanf("%d%d", &n, &m);
    block = sqrt(n);
    for (int i = 1; i <= n; ++i) {
        scanf("%d", &col[i]);
        belong[i] = i / block + 1;
    }
    for (int i = 1; i <= m; ++i) {
        scanf("%d%d", &q[i].l, &q[i].r);
        q[i].id = i;
    }
    sort(q + 1, q + 1 + m, cmp);
    int l = 1, r = 0;
    for (int i = 1; i <= m; ++i) {
        while(r < q[i].r) add(col[++r]);
        while(r > q[i].r) del(col[r--]);
        while(l < q[i].l) del(col[l++]);
        while(l > q[i].l) add(col[--l]);
        res[q[i].id] = ans;
    }
    for (int i = 1; i <= m; ++i) printf("%lld\n", res[i]);
    return 0;
}
\end{lstlisting}
\subsubsection{数位dp计和}
\begin{lstlisting}
#include <bits/stdc++.h>
#define ll long long
using namespace std;
const int mod = 998244353;
pair<ll, ll> dp[20][1<<10];
bool vis[20][1<<10];
int k;
int t[20];
ll base[20];

pair<ll, ll> dfs(int pos, int state, bool limit, bool lead) {
    if (pos == -1) return __builtin_popcount(state) <= k ? make_pair(1, 0) : make_pair(0, 0);
    if (!limit && !lead && vis[pos][state]) return dp[pos][state];
    int up = limit ? t[pos] : 9;
    pair<ll, ll> res = {0, 0};
    for (int i = 0; i <= up; ++i) {
        int n_s = state;
        if (lead && i == 0) n_s = 0;
        else n_s = state | (1 << i);
        auto tmp = dfs(pos - 1, n_s, limit && i == t[pos], lead && i == 0);
        ll pre = 1ll * i * base[pos] % mod;
        (res.first += tmp.first) %= mod;
        (res.second += tmp.second + pre * tmp.first) %= mod;
    }
    if (!limit && !lead) dp[pos][state] = res, vis[pos][state] = 1;
    return res;
}

ll solve(ll x) {
    int pos = 0;
    do {
        t[pos ++] = x % 10;
    } while (x /= 10);
    return dfs(pos - 1, 0, true, true).second;
}

int main(int argc,char *argv[])
{
    base[0] = 1;
    for (int i = 1; i < 20; ++i) base[i] = base[i - 1] * 10;
    ll l, r;
    scanf("%lld%lld%d", &l, &r, &k);
    printf("%lld\n", (solve(r) - solve(l - 1) + mod) % mod);
    return 0;
}
\end{lstlisting}
\subsubsection{相交回文串对数}
\begin{lstlisting}
#include<bits/stdc++.h>

#define ll long long
using namespace std;

const int maxn = 2e6+6;
const int N = 26;
const int mod = 51123987;

struct Palindromic_Tree {
    vector<pair<int, int> > next[maxn];
//    int next[maxn][N];//next指针,next指针和字典树类似,指向的串为当前串两端加上同一个字符构成
    int fail[maxn]{};//fail指针,失配后跳转到fail指针指向的节点
    int cnt[maxn]{}; //表示节点i表示的本质不同的串的个数(建树时求出的不是完全的,最后count()函数跑一遍以后才是正确的)
    int num[maxn]{}; //表示以节点i表示的最长回文串的最右端点为回文串结尾的回文串个数
    int len[maxn]{};//len[i]表示节点i表示的回文串的长度(一个节点表示一个回文串)
    int S[maxn]{};//存放添加的字符
    int last{};//指向新添加一个字母后所形成的最长回文串表示的节点。
    int n{};//表示添加的字符个数。
    int p{};//表示添加的节点个数。

    int newnode(int l) {//新建节点
        next[p].clear();
//        for (int i = 0; i < N; ++i) next[p][i] = 0;
//        cnt[p] = 0;
//        num[p] = 0;
        len[p] = l;
        return p++;
    }

    void init() {//初始化
        n = last = p = 0;
        newnode(0);
        newnode(-1);
        S[n] = -1;//开头放一个字符集中没有的字符,减少特判
        fail[0] = 1;
    }

    int get_fail(int x) {//和KMP一样,失配后找一个尽量最长的
        while (S[n - len[x] - 1] != S[n]) x = fail[x];
        return x;
    }

    int find(int u, int c) {
        vector<pair<int, int> > & x = next[u];
        int sz = x.size();
        for(int i = 0; i < sz; ++i) {
            if(x[i].first == c) return x[i].second;
        }
        return 0;
    }

    int add(int c) {
        S[++n] = c;
        int cur = get_fail(last);//通过上一个回文串找这个回文串的匹配位置
        int x = find(cur, c);
        if (!x) {
//        if (!next[cur][c]) {//如果这个回文串没有出现过,说明出现了一个新的本质不同的回文串
            int now = newnode(len[cur] + 2);//新建节点
            x = now;
            fail[now] = find(get_fail(fail[cur]), c);
            next[cur].emplace_back(make_pair(c, now));
//            fail[now] = next[get_fail(fail[cur])][c];//和AC自动机一样建立fail指针,以便失配后跳转
//            next[cur][c] = now;
            num[now] = num[fail[now]] + 1;
        }
        last = x;
//        last = next[cur][c];
//        cnt[last]++;
        return num[last];
    }

    void count() {
        for (int i = p - 1; i >= 0; --i) cnt[fail[i]] += cnt[i];
        //父亲累加儿子的cnt,因为如果fail[v]=u,则u一定是v的子回文串!
    }
} solve;

char s[maxn];

ll a[maxn], b[maxn];
int main() {
    solve.init();
    int n;
    scanf("%d", &n);
    scanf("%s", s);
    for (int i = 0; i < n; ++i) {
        a[i] = solve.add(s[i] - 'a');
    }
    solve.init();
    for (int i = n - 1; i >= 0; --i) {
        b[i] = (b[i + 1] + solve.add(s[i] - 'a')) % mod;
    }
    ll res = (b[0] * (b[0] - 1) / 2) % mod;
    for (int i = 0; i < n; ++i) {
        res = ((res - (a[i] * b[i + 1]) + mod) % mod) % mod;
    }
    printf("%lld\n", res);
    return 0;
}
\end{lstlisting}
\clearpage\section{秃子的模板}
\subsection{秃子的模板}
\subsubsection{System of Difference Constraints}
\begin{lstlisting}
#include <bits/stdc++.h>

using namespace std;

const int maxn=1000+10;
const int inf=0x3f3f3f3f;

struct Edge {int nex,to,w; } edge[10*maxn];

int head[maxn],cnt,dis[maxn],n;
bool vis[maxn];

void init()
{
	cnt=0;
	memset(head,0xff,sizeof head);
}

void add(int u,int v,int w)
{
	edge[cnt].nex=head[u];
	edge[cnt].to=v;
	edge[cnt].w=w;
	head[u]=++cnt;
}

void spfa(int u)
{
	int u,v,w;
	for(int i=1;i<=n;i++) dis[i]=inf,vis[i]=false;
	dis[u]=0;
	queue<int> que;
	que.push(u);
	vis[u]=true;
	while(!que.empty())
	{
		u=que.front();
		que.pop();
		vis[u]=false;
		for(int i=head[u];~i;i=edge[i].nex)
		{
			v=edge[i].v,w=edge[i].w;
			if(dis[u]+w<dis[v])
			{
				dis[v]=dis[u]+w;
				if(!vis[v])
				{
					que.push(v);
					vis[v]=true;
				}
			}
		}
	}
}\end{lstlisting}
\subsubsection{Aho-Corasick automation}
\begin{lstlisting}
const int maxn=5e5+10;

class AC_automatiion
{
public:
	int trie[maxn][26],cnt;
	int tag[maxn];
	int fail[maxn];

	void init()
	{
	    memset(trie,0,sizeof trie);
	    memset(tag,0,sizeof tag);
	    memset(fail,0,sizeof fail);
	    cnt=0;
	}

	void insert(char *str)
	{
		int root=0;
		for(int i=0;str[i];i++)
		{
			int id=str[i]-'a';
			if(!trie[root][id]) trie[root][id]=++cnt;
			root=trie[root][id];
		}
		tag[root]++;
	}

	void build()
	{
		queue<int> que;
		for(int i=0;i<26;i++) if(trie[0][i]) que.push(trie[0][i]);
		while(!que.empty())
		{
			int k=que.front(); que.pop();
			for(int i=0;i<26;i++)
			{
				if(trie[k][i]) {
					fail[trie[k][i]]=trie[fail[k]][i];
					que.push(trie[k][i]);
				}
				else trie[k][i]=trie[fail[k]][i];
			}
		}
	}

	int query(char *str)
	{
		int p=0,res=0;
		for(int i=0;str[i];i++)
		{
			p=trie[p][str[i]-'a'];
			for(int j=p;j&&~tag[j];j=fail[j]) res+=tag[j],tag[j]=-1;
		}
		return res;
	}
}AC;\end{lstlisting}
\subsubsection{Persistence Trie}
\begin{lstlisting}
#include <bits/stdc++.h>

using namespace std;

const int maxn = 6e5 + 10;

int trie[maxn*24][2], latest[maxn*24];
int s[maxn], root[maxn], n, m, tot;

void insert(int i, int k, int p, int q)
{
	if(k < 0) {
		latest[q] = i;
		 return;
	}
	int c= s[i] >> k & 1;
	if(p) trie[q][c^1] = trie[p][c^1];
	trie[q][c] = ++tot;
	insert(i, k - 1, trie[p][c], trie[q][c]);
	latest[q] = max(latest[trie[q][0]], latest[trie[q][1]]);
}

int ask(int now, int val, int k, int limit)
{
	if(k < 0) return s[latest[now]] ^ val;
	int c = val >> k & 1;
	if(latest[trie[now][c^1]] >= limit) return ask(trie[now][c ^ 1], val, k - 1, limit);
	else return ask(trie[now][c], val, k - 1, limit);
}

int main()
{
    int n, m;
    scanf("%d%d", &n, &m);
    latest[0] = -1;
    root[0] = ++tot;
    insert(0, 23, 0, root[0]);
    for(int i = 1, x; i <= n; i++)
    {
        scanf("%d", &x);
        s[i] = s[i - 1] ^ x;
        root[i] = ++tot;
        insert(i, 23, root[i - 1], root[i]);
    }
    for(int i = 1, x, l, r; i <= m; i++)
    {
        char op[2];
        scanf("%s", op);
        if(op[0] == 'A')
        {
            scanf("%d", &x);
            root[++n] = ++tot;
            s[n] = s[n - 1] ^ x;
            insert(n, 23, root[n - 1], root[n]);
        }
        else
        {
            scanf("%d%d%d", &l, &r, &x);
            printf("%d\n", ask(root[r - 1], x ^ s[n], 23, l - 1));
        }
    }
    return 0;
}
\end{lstlisting}
\subsubsection{Linear Basis}
\begin{lstlisting}
typedef long long ll;

const int MAX_BASE=63;
const int maxn=1e5+10;

int n;
ll a[maxn],b[MAX_BASE+5];

void cal()
{
    for(int i=0;i<n;i++)
    {
        for(int j=MAX_BASE;j>=0;j--)
        {
            if(a[i]>>j&1)
            {
                if(b[j]) a[i]^=b[j];
                else
                {
                    b[j]=a[i];
                    for(int k=j-1;k>=0;k--) if(b[k]&&(b[j]>>k&1)) b[j]^=b[k];
                    for(int k=j+1;k<=MAX_BASE;k++) if(b[k]>>j&1) b[k]^=b[j];
                    break;
                }
            }
        }
    }
}
\end{lstlisting}
\subsubsection{MincostMaxflow}
\begin{lstlisting}
#include <vector>
#include <queue>

using namespace std;

const int maxn=10000+10;
const int inf=0x3f3f3f3f;

struct Edge { int from, to, cap, flow, cost; };

struct MCMF
{
	int n,m;
	vector<Edge> edges;
	vector<int> G[maxn];
	bool inq[maxn];
	int dis[maxn], path[maxn], a[maxn];

	void init(int n)
	{
		this->n=n;
		for(int i=0;i<=n;i++)
			G[i].clear();
		edges.clear();
	}

	void addEdge(int from, int to, int cap, int cost)
	{
		edges.push_back(Edge{from,to,cap,0,cost});
		edges.push_back(Edge{to,from,0,0,-cost});
		m=edges.size();
		G[from].push_back(m-2);
		G[to].push_back(m-1);
	}

	bool Bellman_Ford(int s, int t, int& flow, int& cost)
	{
		for(int i=0; i<=n; i++) dis[i]=inf;
		memset(inq, 0, sizeof inq);
		dis[s]=0, inq[s]=true, path[s]=0, a[s]=inf;
		queue<int> Q;
		Q.push(s);
		while(!Q.empty())
		{
			int u=Q.front(); Q.pop();
			inq[u]=false;
			for(int i=0;i<G[u].size();i++)
			{
				Edge& e=edges[G[u][i]];
				if(e.cap>e.flow&&dis[e.to]>dis[u]+e.cost)
				{
					dis[e.to]=dis[u]+e.cost;
					path[e.to]=G[u][i];
					a[e.to]=min(a[u],e.cap-e.flow);
					if(!inq[e.to])
					{
						Q.push(e.to);
						inq[e.to]=true;
					}
				}
			}
		}
		if(dis[t]==inf) return false;
		flow+=a[t];
		cost+=dis[t]*a[t];
		for(int u=t;u!=s;u=edges[path[u]].from)
		{
			edges[path[u]].flow+=a[t];
			edges[path[u]^1].flow-=a[t];
		}
		return true;
	}

	int mincostMaxFlow(int s, int t, int& cost)
	{
		int flow=0;
		cost=0;
		while(Bellman_Ford(s,t,flow,cost));
		return flow;
	}
};
\end{lstlisting}
\subsubsection{Suffix Array}
\begin{lstlisting}
const int maxn=1e5+10;

char s[maxn];
int sa[maxn],t[maxn],t2[maxn],c[maxn],n;
int ra[maxn],height[maxn];

void build_sa(int m)
{
	int *x=t,*y=t2;
	for(int i=0;i<m;i++) c[i]=0;
	for(int i=0;i<n;i++) c[x[i]=s[i]]++;
	for(int i=1;i<m;i++) c[i]+=c[i-1];
	for(int i=n-1;i>=0;i--) sa[--c[x[i]]]=i;
	for(int k=1;k<=n;k<<=1)
	{
		int p=0;
		for(int i=n-k;i<n;i++) y[p++]=i;
		for(int i=0;i<n;i++) if(sa[i]>=k) y[p++]=sa[i]-k;
		for(int i=0;i<m;i++) c[i]=0;
		for(int i=0;i<n;i++) c[x[y[i]]]++;
		for(int i=0;i<m;i++) c[i]+=c[i-1];
		for(int i=n-1;i>=0;i--) sa[--c[x[y[i]]]]=y[i];
		swap(x,y);
		p=1;x[sa[0]]=0;
		for(int i=1;i<n;i++)
			x[sa[i]]=y[sa[i-1]]==y[sa[i]]&&y[sa[i-1]+k]==y[sa[i]+k]?p-1:p++;
		if(p>=n) break;
		m=p;
	}
}

void getHeight()
{
    for(int i=0;i<n;i++) ra[sa[i]]=i;
    for(int i=0,k=0;i<n;i++)
    {
        if(k) k--;
        int j=sa[ra[i]-1];
        while(i+k<n&&j+k<n&&s[i+k]==s[j+k]) k++;
        height[ra[i]]=k;
    }
}
\end{lstlisting}
\clearpage\section{hld and more}
\subsection{hld}
\begin{lstlisting}
/*
sz[]数组,以x为根的子树节点个数
top[]数组,当前节点的所在链的顶端节点
son[]数组,重儿子
deep[]数组,当前节点的深度
fa[]数组,当前节点的父亲
idx[]数组,树中每个节点剖分后的新编号
rnk[]数组,idx的逆,表示线段上中当前位置表示哪个节点
*/
struct HLD {
#define type int

    struct edge {
        int a, b;
        type v;

        edge(int _a, int _b, type _v = 0) : a(_a), b(_b), v(_v) {}
    };

    struct node {
        int to;
        type w;

        node() {}

        node(int _to, type _w) : to(_to), w(_w) {}
    };

    vector<int> mp[maxn];
    vector<edge> e;

    void init(int _n) {
        n = _n;
        for (int i = 0; i <= n; i++) mp[i].clear();
        e.clear();
        e.push_back(edge(0, 0));
    }

    void add_edge(int a, int b, type v = 0) {
//        e.push_back(edge(a,b,v));
        mp[a].push_back(b);
        mp[b].push_back(a);
    }

    void dfs1(int x, int pre, int h) {
        int i, to;
        deep[x] = h;
        fa[x] = pre;
        sz[x] = 1;
        for (i = 0; i < (int) (mp[x].size()); i++) {
            to = mp[x][i];
            if (to == pre) continue;
            dfs1(to, x, h + 1);
            sz[x] += sz[to];
            if (son[x] == -1 || sz[to] > sz[son[x]]) son[x] = to;
        }
    }

    void dfs2(int x, int tp) {
        int i, to;
        top[x] = tp;
        idx[x] = ++tot;
        rnk[idx[x]] = x;
        if (son[x] == -1) return;
        dfs2(son[x], tp);
        for (i = 0; i < (int) (mp[x].size()); i++) {
            to = mp[x][i];
            if (to != son[x] && to != fa[x]) dfs2(to, to);
        }
    }

    void work(int _rt = 1) {
        memset(son, -1, sizeof son);
        tot = 0;
        dfs1(_rt, 0, 0);
        dfs2(_rt, _rt);
    }

    int LCA(int x, int y) {
        while (top[x] != top[y]) {
            if (deep[top[x]] < deep[top[y]]) swap(x, y);
            x = fa[top[x]];
        }
        if (deep[x] > deep[y]) swap(x, y);
        return x;
    }

    void modify_node(int x, int y, type val) {
        while (top[x] != top[y]) {
            if (deep[top[x]] < deep[top[y]]) swap(x, y);
            le = idx[top[x]], re = idx[x];
            k = val;
            update(1, 1, n);
            x = fa[top[x]];
        }
        if (deep[x] > deep[y]) swap(x, y);
        le = idx[x], re = idx[y];
        k = val;
        update(1, 1, n);
    }

    type query_node(int x, int y) {
        type res = 0;
        while (top[x] != top[y]) {
            if (deep[top[x]] < deep[top[y]]) swap(x, y);
            le = idx[top[x]], re = idx[x];
            res += query(1, 1, n);
            x = fa[top[x]];
        }
        if (deep[x] > deep[y]) swap(x, y);
        le = idx[x], re = idx[y];
        res += query(1, 1, n);
        return res;
    }

    //path
//    void init_path()
//    {
//        v[idx[rt]]=0;
//        for(int i=1;i<n;i++)
//        {
//            if(deep[e[i].a]<deep[e[i].b]) swap(e[i].a,e[i].b);
//            a[idx[e[i].a]]=e[i].v;
//        }
//        build(n);
//    }
    void modify_edge(int id, type val) {
        if (deep[e[id].a] > deep[e[id].b]) {
            le = idx[e[id].a], re = idx[e[id].a];
            k = val;
            update(1, 1, n);
        } else {
            le = idx[e[id].b], re = idx[e[id].b];
            k = val;
            update(1, 1, n);
        }
    }

    void modify_path(int x, int y, type val) {
        while (top[x] != top[y]) {
            if (deep[top[x]] < deep[top[y]]) swap(x, y);
            le = idx[top[x]], re = idx[x];
            k = val;
            update(1, 1, n);
            x = fa[top[x]];
        }
        if (deep[x] > deep[y]) swap(x, y);
        if (x != y) {
            le = idx[x] + 1, re = idx[y];
            k = val;
            update(1, 1, n);
        }
    }

    type query_path(int x, int y) {
        type res = 0;
        while (top[x] != top[y]) {
            if (deep[top[x]] < deep[top[y]]) swap(x, y);
            le = idx[top[x]], re = idx[x];
            res += query(1, 1, n);
            x = fa[top[x]];
        }
        if (deep[x] > deep[y]) swap(x, y);
        if (x != y) {
            le = idx[x] + 1, re = idx[y];
            res += query(1, 1, n);
        }
        return res;
    }

#undef type
} hld;
\end{lstlisting}
\subsection{dijkstra}
\begin{lstlisting}
const int maxn=1e5+10;
const int inf=2147483647;

int head[maxn], dis[maxn], cnt, n;

struct Edge { int nex,to,w; }edge[20*maxn];

void add(int u,int v,int w)
{
    edge[++cnt].nex=head[u];
    edge[cnt].w=w;
    edge[cnt].to=v;
    head[u]=cnt;
}

void dijkstra(int s)
{
    priority_queue<pair<int, int> vector<pair<int, int> >, greater<pair<int, int> > > que;
    memset(dis, 0x3f, sizeof dis);
    que.push({0, s}); dis[s] = 0;
    while(!que.empty())
    {
        aoto f = que.top(); que.pop();
        int u = f.second, d = f.first;
        if(d != dis[u]) continue;
        for(int i = head[u]; ~i; i = edge[i].nex)
        {
            int v = edge[i].to, w = edge[i].w;
            if(dis[u] + w < dis[v])
            {
                dis[v] = dis[u] + w;
                que.push({dis[v], v});
            }
        }
    }
}
\end{lstlisting}

\end{document}
